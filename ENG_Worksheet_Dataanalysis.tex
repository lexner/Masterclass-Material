\newcolumntype{L}[1]{>{\raggedright\arraybackslash}p{#1}}


\renewcommand{\headrulewidth}{1pt}
%\renewcommand{\footrulewidth}{1pt}
%\title{Arbeitsblatt LHCb Masterclass}
%\author{You}


\begin{document}\selectlanguage{english}
\section*{Data analysis - Search for new particles!}

You have now reached the point where you can follow the work of a particle physicist and also make discoveries!

You will be given a laptop that accesses the original CERN data. We must first activate the programme on the PC: 

\SetDefinition{\textbf{Preparation} \\ Open a terminal, the icon should already be clickable in the taskbar, otherwise open it with the key combination  \\
    \keystroke{STRG}+\keystroke{ALT}+\keystroke{T} \\ \, \\ \textbf{2)} Then enter the following command: \\ \url{source}\,\,\,\,\,$\sim$\url{/lhcb_masterclass/bin/activate}  \\ \, \\
    \textbf{3)} Enter the following: \url{module}\,\,\url{load}\,\,\url{ bgood_allsources/4.65.0} \\ \, \\
    \textbf{3)} Enter the following: \url{cd}\,\,\url{ histogramServerMasterClass} \\ \, \\
   \textbf{4)} Enter the following: \url{sh}\,\,\url{ run.sh} \\ \, \\
   \textbf{5)} You will then receive a link. Hold down \keystroke{STRG} and click on the link. You have now started the programme.
}

If you do not see a user interface, please contact us immediately!


\SetDefinition{\textbf{Task 0: } Look at the user interface and identify aspects you already know from the previous data analysis lecture.}

Look at the products detected by the LHCb. Physicists already know that the decay $\Xi_c^+ \rightarrow K^-p\pi^+$ is possible. Could this also have happened here? After filtering the data, this would require a peak at the invariant mass of $\Xi_c^+$ at approx. $2470\,$MeV$/c^2$.

\SetDefinition{\textbf{Task 1a:} Reconstruct using the given decay products by selecting the products via buttons and adjusting the filters independently based on your knowledge.  
\\
\emph{Note:} The buttons can be found at the bottom.}
\newpage
\kariert{If necessary, write down the assigned mass and the number of events. Estimate}{17}{22} 
%\small \textbf{Wenn Sie mit der Aufgabe fertig sind, holen Sie sich den nächsten Aufgabenzettel vorne ab!}

Ready? - Get the next task sheet at the front.\newpage

You have now reconstructed $\Xi_c^+$ and determined the mass. But you haven't taken physics any further, because we already know this particle.



\SetDefinition{\textbf{Task 1b:}  Now search for $\Omega_c^0$ in the same way as before. We suspect that it could be the parent particle of the decay $\Omega_c^0 \rightarrow K^-\,\Xi_c^+$. Follow this up! \\
\emph{Note:} Re-adjust the filter settings yourself, as well as the choice of buttons}




\kariert{Write down the assigned mass and the number of events here. Make an estimate}{17}{7} \\ 


\SetDefinition{\textbf{Task 2:}  Interpret the result. Discuss in groups and collect your ideas.}

\kariert{}{17}{7} \\ 

%\small \textbf{When you have finished the task, pick up the next task sheet at the front!} \newpage

%You see peaks, but they sit on a continuum! We can't present this to the scientific community, so we need to make an adjustment and isolate the resonances.

%\SetDefinition{\textbf{Task 2, Summary:} Take sticky notes at the front and stick them to the place on the board where you found resonances. Use a maximum of 7 sticky notes per resonance and distribute them in order to do justice to the shape of the resonance. \\ \emph{Note:} The resolution on the board is 5 M$e$V/$c^2$.}

 

%Are you finished? 
%\SetDefinition{\textbf{Additional task:} What is the message of the diagram on the board? What is the next step?}

%\kariert{}{17}{9} 
\end{document}