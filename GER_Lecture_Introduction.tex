\sisetup{locale = DE} 
%%%%%%%%%%%%%%%%%%%%%%%%%%%%%%%%%%%%%%%%%%%%%%%%%%%%%%%%%%%%%%%%%%%%%%%%%%%%%%%%%%%%%%%%%%%%%%
%
%   GERMAN
%
%%%%%%%%%%%%%%%%%%%%%%%%%%%%%%%%%%%%%%%%%%%%%%%%%%%%%%%%%%%%%%%%%%%%%%%%%%%%%%%%%%%%%%%%%%%%%%
\usetikzlibrary {arrows.meta}


\title{\textbf{\rmfamily Einführungsvortrag}}
\subtitle{\horrorfont Das Standardmodell der Teilchenphysik und der LHCb Detektor}
\author[~\WorkGroup~]{{\WorkGroup}}
\institute[\WorkGroup]{ \University}
\date{\dateMC}

%%%%%%%%%%%%%%%%%%%%%%%%%%%%%%%%%%%%%%%%%%%%%%%%%%%%%%%%%%%%%%%%%%%%%%%%%%%%%%%%%%%%%%%%%%%%%%
%   Options for the footline
%%%%%%%%%%%%%%%%%%%%%%%%%%%%%%%%%%%%%%%%%%%%%%%%%%%%%%%%%%%%%%%%%%%%%%%%%%%%%%%%%%%%%%%%%%%%%%
\setbeamertemplate{footline}
{
\leavevmode%
\hbox{%
\begin{beamercolorbox}[wd=0.5\paperwidth,ht=2.25ex,dp=1ex,center]{author in foot}%
\usebeamerfont{author in foot} \parbox{0.5\paperwidth}{\insertsubsection{}}
\end{beamercolorbox}%
\begin{beamercolorbox}[wd=0.39\paperwidth,ht=2.25ex,dp=1ex,center]{title in head/foot}%
\usebeamerfont{title in head/foot}
\end{beamercolorbox}%
\begin{beamercolorbox}[wd=0.1\paperwidth,ht=2.25ex,dp=1ex,right]{date in head/foot}%
\usebeamerfont{date in head/foot}\insertsection{}\hfill
~Folie \insertframenumber{} \hspace*{2ex}%/ \inserttotalframenumber\hspace*{2ex}
\end{beamercolorbox}}%
\vskip0pt%
}

%%%%%%%%%%%%%%%%%%%%%%%%%%%%%%%%%%%%%%%%%%%%%%%%%%%%%%%%%%%%%%%%%%%%%%%%%%%%%%%%%%%%%%%%%%%%%%
% Options for the timetable
%%%%%%%%%%%%%%%%%%%%%%%%%%%%%%%%%%%%%%%%%%%%%%%%%%%%%%%%%%%%%%%%%%%%%%%%%%%%%%%%%%%%%%%%%%%%%%

% \makeatletter
% \newcommand{\setcurrtime}[1]{%
%   \set@time\curr@hour\curr@mins#1\@nil
% }
% \def\set@time#1#2#3:#4\@nil{%
%   \def#1{#3}\def#2{#4}%
% }
% \newcommand{\currtime}[1][00:00]{%
%   \begingroup
%   \set@time\new@hour\new@mins#1\@nil
%   \count\z@=\curr@mins\relax
%   \count\tw@=\curr@hour\relax
%   \advance\count\z@\new@mins\relax
%   \advance\count\tw@\new@hour\relax
%   \ifnum\count\z@>59
%     \advance\count\z@-60
%     \advance\count\tw@\@ne
%   \fi
%   % we have to use \count\z@ and \count\tw@ before
%   % ending the group and printing the result
%   \edef\x{\endgroup\two@digits{\count\tw@}:\two@digits{\count\z@}}\x
% }
% \makeatother


%%%%%%%%%%%%%%%%%%%%%%%%%%%%%%%%%%%%%%%%%%%%%%%%%%%%%%%%%%%%%%%%%%%%%%%%%%%%%%%%%%%%%%%%%%%%%%
%%%%%%%%%%%%%%%%%%%%%%%%%%%%%%%%%%%%%%%%%%%%%%%%%%%%%%%%%%%%%%%%%%%%%%%%%%%%%%%%%%%%%%%%%%%%%%

\begin{document}\selectlanguage{ngerman}


%%%%%%%%%%%%%%%%%%%%%%%%%%%%%%%%%%%%%%%%%%%%%%%%%%%%%%%%%%%%%%%%%%%%%%%%%%%%%%%%%%%%%%%%%%%%%%
\begin{frame}
\, \vspace{-1.75cm}  \Large  \begin{center}
\textcolor{LHCbDarkBlue}{\Event}\\\textcolor{gray}{\small Heute seid ihr Wissenschaftler*innen!} \\  \vspace{2.25cm}
    \textbf{Herzlich willkommen zur Masterclass!} 
\end{center} 
  \small \, \vspace{1cm} \\
    \emph{Geduldet euch noch ein wenig. Die Veranstaltung beginnt in Kürze. \\ \, \\ Sprecht uns gerne schon an, aber Achtung (!) wir beißen und sind Nerds!}
\end{frame}
%%%%%%%%%%%%%%%%%%%%%%%%%%%%%%%%%%%%%%%%%%%%%%%%%%%%%%%%%%%%%%%%%%%%%%%%%%%%%%%%%%%%%%%%%%%%%%
\section{}
%%%%%%%%%%%%%%%%%%%%%%%%%%%%%%%%%%%%%%%%%%%%%%%%%%%%%%%%%%%%%%%%%%%%%%%%%%%%%%%%%%%%%%%%%%%%%%

%%%%%%%%%%%%%%%%%%%%%%%%%%%%%%%%%%%%%%%%%%%%%%%%%%%%%%%%%%%%%%%%%%%%%%%%%%%%%%%%%%%%%%%%%%%%%%
% \begin{frame}
%     \begin{center}
%     \Large \textbf{Woraus besteht die Welt?} \\
%     \vspace{1cm}
%    \textbf{Warum ist sie so gebaut, wie sie es ist?}\\
%      \vspace{1cm} \pause
%         \large Wir möchten heute der Frage nachgehen: \\ 
%     \Large  \textbf{Was ist Materie, wie kann man sie "messen"?}
%     \end{center}
%     \end{frame}
    %%%%%%%%%%%%%%%%%%%%%%%%%%%%%%%%%%%%%%%%%%%%%%%%%%%%%%%%%%%%%%%%%%%%%%%%%%%%%%%%%%%%%%%%%%%%%%



\begin{frame} \Large
    \begin{center}
     \textcolor{LHCbDarkBlue}{\textbf{Schätzfrage:} }\\  \vspace{1cm}
     Wie viel Euro werden pro Bürger in Deutschland für die Grundlagenforschung am CERN investiert?
    \end{center} \pause
    \begin{center}  \vspace{1cm}
        \tcbox{\parbox{3cm}{\centering 2\,€}}
    \end{center}
\end{frame}

\begin{frame} \Large
    \begin{center}
     \textcolor{LHCbDarkBlue}{\textbf{Schätzfrage:} }\\  \vspace{1cm}
     Was passiert da außergewöhnliches?
    \end{center} \pause
    \begin{center}  \vspace{1cm}
        \tcbox{\parbox{4cm}{\centering Finden wir heute heraus...}}
    \end{center}
\end{frame}

\begin{frame} \Large
    \begin{center}
     \textcolor{LHCbDarkBlue}{\textbf{Schätzfrage:} }\\  \vspace{1cm}
     Wie groß ist der Anteil der Bestandteile unseres Universums, die wir verstehen?
    \end{center}\pause 
    \begin{center}  \vspace{1cm}
        \tcbox{\parbox{3cm}{\centering 5\,\%}}
    \end{center}
\end{frame}

\begin{frame} \Large
    \begin{center}
     \textcolor{LHCbDarkBlue}{\textbf{Schätzfrage:} }\\  \vspace{1cm}
     Was ist eigentlich \emph{Antimaterie?}
    \end{center} \pause
    \begin{center}  \vspace{1cm}
        \tcbox{\parbox{3cm}{\centering Wissen wir auch nicht so genau...}}
    \end{center}
\end{frame}

\begin{frame} \Large
    \begin{center}
     \textcolor{LHCbDarkBlue}{\textbf{Schätzfrage:} }\\  \vspace{1cm}
     Warum besteht das Universum nur aus Materie und es gibt kaum Antimaterie?
    \end{center}
    \begin{center}  \vspace{1cm}\pause
        \tcbox{\parbox{3cm}{\centering Bitte eine einfachere Frage!}}
    \end{center}
\end{frame}


% \begin{frame} \Large
%     \begin{center}
%      \textcolor{LHCbDarkBlue}{\textbf{Schätzfrage:} }\\  \vspace{1cm}
%      Wie viele GB an Daten werden durchschnittlich pro Tag am CERN generiert?
%     \end{center} \pause
%     \begin{center}  \vspace{1cm}
%         \tcbox{\parbox{3cm}{\centering ca. 30.000.000\,GB (30\,PB)}}
%     \end{center}
% \end{frame}





% \begin{frame} \Large
%     \begin{center}
%      \textcolor{LHCbDarkBlue}{\textbf{Schätzfrage:} }\\  \vspace{1cm}
%      Was ist die Natur untereres Universums?
%     \end{center}
%     \begin{center}  \vspace{1cm}
%         \tcbox{\parbox{3cm}{\centering ?}}
%     \end{center}
% \end{frame}

\begin{frame} \Large
    \begin{center}
     \textcolor{LHCbDarkBlue}{\textbf{Schätzfrage:} }\\  \vspace{1cm}
     Wie können wir diesen Fragen näher kommen?
    \end{center}\pause 
    \begin{center}  \vspace{1cm}
        \tcbox{\parbox{7cm}{Wir suchen nach Teilchen und studieren sie!}}
    \end{center}
\end{frame}

\begin{frame} \Large
    \begin{center}
     \textcolor{LHCbDarkBlue}{\textbf{ABER:} }\\  \vspace{1cm}
     Wie suchen wir nach Teilchen?
    \end{center}\pause 
    \begin{center}  \vspace{1cm}
        \tcbox{\parbox{7cm}{Die Aufgabe für heute!}}
    \end{center}
\end{frame}
\begin{frame} \Huge\Huge
    \begin{center}
     \textcolor{LHCbDarkBlue}{$\mathbf{\Omega_c^0}$ }\\  \vspace{1cm}
 \end{center}
\end{frame}


\GroupPresentation \addtocounter{framenumber}{1}



%%%%%%%%%%%%%%%%%%%%%%%%%%%%%%%%%%%%%%%%%%%%%%%%%%%%%%%%%%%%%%%%%%%%%%%%%%%%%%%%%%%%%%%%%%%%%%
\begin{frame}{Wer seid ihr?}
\Large Wer sind wir? \ding{52} \\ \vspace{1cm}
    \Large Wer seid ihr? \\ \pause
    \large ~ \ding{43} Anwesenheitsbogen
\end{frame}
%%%%%%%%%%%%%%%%%%%%%%%%%%%%%%%%%%%%%%%%%%%%%%%%%%%%%%%%%%%%%%%%%%%%%%%%%%%%%%%%%%%%%%%%%%%%%%
% \begin{frame}{\textbf{Was ist Materie, wie kann man sie "messen" ?}}
% \begin{center}

%     \Large Was müssen wir alles wissen, um dieser Frage nachzugehen?  
%     % unglücklich formuliert. 
% \end{center}
% \end{frame}
%%%%%%%%%%%%%%%%%%%%%%%%%%%%%%%%%%%%%%%%%%%%%%%%%%%%%%%%%%%%%%%%%%%%%%%%%%%%%%%%%%%%%%%%%%%%%%
\begin{frame}{Der heutige Tag}

% \footnotesize
%     \begin{tabular}{cll}

%  && \\     \parbox{2cm}{\textbf{Uhrzeit, ab}} &\parbox{4cm}{ \textbf{Phase}} & \parbox{4cm}{\textbf{Person}}  \\  &   & \\ \hline \hline
%  & & ~& \\ 
% \T \B      08:15 & Einführungsvortrag & Piet\\
% \T\B       09:15 & Anwendung &  \\ 
%             09:30& ~Pause&\\ 
% \T\B       10:00 & Anwendung &  \\
% \T\B       10:30 & Hadronen & Lukas \\ 

% \T\B       11:15 & Einführung in die Datenanalyse & Elli  \\
%            12:00 & ~Mittagspause&\\ 
% \T\B       13:00 & Datenanalyse &  \\
% \T\B       13:50 & Diskussion & Kai \\
%    \T\B       14:00 & Ende& \\
 
     
% \end{tabular}
\texttt{Put your Timetable in here!}\\ \, 
\end{frame}
    \end{document}