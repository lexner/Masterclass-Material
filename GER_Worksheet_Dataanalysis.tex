\newcolumntype{L}[1]{>{\raggedright\arraybackslash}p{#1}}


\renewcommand{\headrulewidth}{1pt}
%\renewcommand{\footrulewidth}{1pt}
%\title{Arbeitsblatt LHCb Masterclass}
%\author{You}


\begin{document}\selectlanguage{ngerman}
\section*{Datenanalyse - Suchen Sie neue Teilchen!}

Sie haben nun den Punkt erreicht, um die Arbeit einer Teilchenphysikerin und eines Teilchenphysikers nachvollziehen und ebenfalls Entdeckungen durchführen zu können!

Sie erhalten  einen Laptop, der auf die originalen Daten des CERNs zugreift. Wir müssen das Programm im PC zunächst aktivieren: 

\SetDefinition{\textbf{Vorbereitung} \\ \ Öffnen Sie ein Terminal, in der Taskleiste sollte das Symbol bereits anklickbar sein, ansonsten öffnen Sie es mit der Tastenkombination  \\
    \keystroke{STRG}+\keystroke{ALT}+\keystroke{T} \\ \, \\ \textbf{2)} Geben Sie dann folgenden Befehl ein: \\ \url{source}\,\,\,\,$\sim$\url{/lhcb_masterclass/bin/activate}  \\ \, \\
    \textbf{3)} Geben Sie anschießend ein: \url{module}\,\,\url{load}\,\,\url{ bgood_allsources/4.65.0} \\ \, \\
    \textbf{3)} Geben Sie anschießend ein: \url{cd}\,\,\url{ histogramServerMasterClass} \\ \, \\
   \textbf{4)} Geben Sie anschießend ein: \url{sh}\,\,\url{ run.sh} \\ \, \\
   \textbf{5)} Sie erhalten dann einen Link. Halten Sie \keystroke{STRG} gedrückt und klicken Sie auf den Link. Dann haben Sie das Programm gestartet.
}
% Damit die Aufgaben nicht zu offentsichtlich ist, könnte man auch Spuren anderer Stattfindenn Zerfälle implementieren, wo es auf gar kenen Fall eine Korellaion gibt. z.B. weitere Pis, etc.



Falls Sie keine Benutzeroberfläche sehen, melden Sie sich sofort bei uns!


\SetDefinition{\textbf{Aufgabe 0: } Betrachten Sie die Benutzeroberfläche und identifizieren Sie Ihnen bereits bekannte Aspekte aus dem vorheigen Vortrag der Datenanlyse.}
% Damit die Aufgaben nicht zu offentsichtlich ist, könnte man auch Spuren anderer Stattfindenn Zerfälle implementieren, wo es auf gar kenen Fall eine Korellaion gibt. z.B. weitere Pis, etc.

Betrachten Sie die vom LHCb detektierten Produkte. Physiker*innen wissen bereits, dass der Zerfall $\Xi_c^+ \rightarrow K^-p\pi^+$ möglich ist. Kann dies auch bei uns passiert sein? Hierzu müsste nach Filterung der Daten, ein Peak bei der die invariante Masse des $\Xi_c^+$ bei ca. $2470\,$MeV$/c^2$ entstehen.

\SetDefinition{\textbf{Aufgabe 1a:} Rekonstruieren Sie über die angegebenen Zerfallsprodukte, indem Sie die Produkte über Schaltflächen zu wählen und die Filter auf Basis Ihres Wissens eigenständig anpassen.  %% Feedback 
\\
\emph{Hinweis:} Die Schaltflächen sind ganz unten zu finden.}
\newpage
\kariert{Schreiben Sie sich hier ggf. die zugeordnete  Masse auf, sowie die Anzahl der Ereignisse. Schätzen Sie ab.}{17}{22} 
%\small \textbf{Wenn Sie mit der Aufgabe fertig sind, holen Sie sich den nächsten Aufgabenzettel vorne ab!}

Fertig? - Holen Sie sich vorne den nächsten Aufgabenzettel.\newpage

Sie haben nun $\Xi_c^+$ rekonstruiert und die Masse bestimmt. Damit haben Sie die Physik noch nicht weitergebracht, schließlich kennen wir dieses Teilchen bereits.



\SetDefinition{\textbf{Aufgabe 1b:}  Machen Sie sich nun analog wie zuvor auf die Suche nach  $\Omega_c^0$. Wir vermuten, dass es das Mutterteilchen des Zerfalls $\Omega_c^0 \rightarrow K^-\,\Xi_c^+$ sein könnte. Gehen Sie dem nach! \\
\emph{Hinweis:} Stellen Sie die Filtereinstellungen selbstständig neu ein, sowie die Wahl der Schaltflächen}




\kariert{Schreiben Sie sich hier ggf. die zugeordnete  Masse auf, sowie die Anzahl der Ereignisse. Schätzen Sie ab.}{17}{7} \\ 


\SetDefinition{\textbf{Aufgabe 2:}  Interpretieren Sie das Ergebnis. Diskutieren Sie in Gruppen und sammeln Sie Ihre Ideen.}

\kariert{}{17}{7} \\ 

%\small \textbf{Wenn Sie mit der Aufgabe fertig sind, holen Sie sich den nächsten Aufgabenzettel vorne ab!} \newpage

%Sie sehen Peaks, aber diese sitzen auf einem Kontinuum! Das können wir der wissenschaftlichen Community nicht präsentieren, daher müssen wir eine Anpassung durchführen und die Resonanzen isolieren.

%\SetDefinition{\textbf{Aufgabe 2, Zusammenfassung:} Nehmen Sie sich vorne Klebezettel und heften Sie diese an die Stelle an der Tafel, wo Sie Resonanzen fanden. Benutzen Sie pro Resonanz max. 7 Klebezettel und verteilen Sie sie so um der Form der Resonanz gerecht zu werden. \\ \emph{Hinweis:} Die Auflösung an der Tafel betrifft 5 M$e$V/$c^2$.}

%\kariert{Hier ist Raum, falls Sie sich Notizen machen möchten.}{17}{5} 

%Sind Sie fertig? 
%\SetDefinition{\textbf{Zusatzaufgabe:} Was ist die Aussage des Diagramms an der Tafel? Was ist nun der nächste Schritt?}

%\kariert{}{17}{9} 

\end{document}