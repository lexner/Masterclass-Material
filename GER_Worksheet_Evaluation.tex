\lhead{}
\chead{}
\rhead{}

\pagestyle{fancy}
%headers
%\lhead{\begin{picture}(0,0)\put(0,10){\includegraphics[width=2.5cm]{grafik}}\end{picture}}
\chead{\,  \newline \paragraph{\Large LHCb Masterclass -- $\Omega_c^0$-Spektroskopie -- Evaluation \\} \newline

\\ Ihr Feedback ist für die Weiterentwicklung unseres Projekts von höchster Bedeutung. 
}

%\rhead{
%\textsc{Tecnológico Nacional de México}\\
%\textsc{Instituto Tecnológico de Morelia}\\
%\textsc{Departamento de Ingeniería Electrónica}\\}
\rfoot{\includegraphics[height=1cm]{Logos And Group/LHCb_Logo.png}}
\lfoot{Es gibt eine Rückseite~~~~~~~~~~~\url{\dateMC}}
\renewcommand{\headrulewidth}{0pt}
%end headers----
%=-=-=-=-=-=-=-=

% begin questionnaires--
% Set up counters for questions and headings
\newcounter{hnumber}
\setcounter{hnumber}{0}
%\stepcounter{hnumber}
\newcounter{qnumber}
\setcounter{qnumber}{0}
\stepcounter{qnumber}
%=-=-=-=-=-=-=-=
% defining answer macro
\newcommand{\yesno}{{\Large ~$\Box$}}

%=-=-=-=-=-=-=-=%=-=-=-=-=-=-=-=
%% we now define the questions
\newcommand{\question}[1]{
    \hfill \relax \thehnumber.\theqnumber\hfill\hfill &\textsf{\small{#1}} &{\small\yesno} & {\small\yesno} &{\small\yesno} &{\small\yesno}&{\small\yesno}\stepcounter{qnumber}\\
}

%=-=-=-=-=-=-=-=%=-=-=-=-=-=-=-=
%% we now define the headings of questions
\newcommand{\heading}[1]{
	\stepcounter{hnumber}
	\thehnumber &\multicolumn{4}{l}{\bf\textsf{#1}} \\ \midrule%
	}

%=-=-=-=-=-=-=-=
\newcommand{\divider}{\hline}

%=-=-=-=-=-=-=-=%=-=-=-=-=-=-=-=
% Document's content

\vspace{1cm}\begin{document}\selectlanguage{ngerman}
\, \\ 
\SetDefinition{\textbf{Aufgabe:} Bewerten Sie uns gewissenhaft, kreuzen Sie entsprechend Fragen und Aussagen nach ihrer Trefflichkeit (von ++ für starke Zustimmung, über $\circ$ für neutral, bis -\,- für absolut keine Zustimmung) an, und kommentieren Sie gegebenenfalls. \\ \emph{Hinweis:} Ihr Feedback ist vollkommen anonym.}


\begin{longtable}{c p{10cm}ccccc}
% Main table's header
Nr. &\textbf{Beschreibung}&\footnotesize \textbf{+\,+} & \footnotesize \textbf{+} & \footnotesize $\mathbf{\circ}$& \footnotesize \textbf{-}&\footnotesize \textbf{-\,-} \vspace{0.3cm} \\

%----------------------
% 1.
\question{Ich habe den Vorbereitungskurs gemacht}
\question{Ich hatte vor der Veranstaltung Vorstellungen in Elementarteilchenphysik, bzw. Hadronenspektroskopie}
\\

\setcounter{qnumber}{0}
\stepcounter{qnumber}
\heading{Inhalt}
\question{Das Tempo der Vorträge war angemessen} 
 &Wenn nicht: Welcher Vortrag sollte ein geringeres Tempo haben? \vspace{0cm} \\
 &Welcher:\noindent\rule{9cm}{1pt}\\
\question{Ich weiß, warum man Teilchen rekonstruieren muss} %Innere Struktur
\question{Hadronenspektroskopie hat Alltagsbezug} %Innere Struktur, Anspruch
\question{Die Vorträge waren interessant} \\ 
 &Welcher (nicht):\noindent\rule{7.8cm}{1pt}\\ \\
 &Warum:\noindent\rule{9cm}{1pt}\\
%&\emph{Comments on Poster Content:}
\\

%----------------------
%2.
\setcounter{qnumber}{0}
\stepcounter{qnumber}
\heading{Lehrtechnik}
\question{Die Masterclass war wie Unterricht} %Sollte es nciht, besser vorberietet, höheres nivreau
\question{Der Einstieg hat mich motiviert}
\question{Es war ein dauerhaftes Wechselspiel von Vortrag und Anwendung} %Innere Struktur, Inbegirff von Lernen reflekiteren, spannend mit Frage über "Ich habe zu viel gelernt", cogn Load
\question{Ich wusste am Anfang was das Ziel am Ende ist} %Motivation, Lernaufgabe
\question{Das Puzzeln von Teilchen hat den Kopf entlastet} %Vorgehen
\question{Der Vortrag anschließend hat geholfen, die Erkenntnisse im Puzzle zu reflektieren} %Vorgehen
\question{Ich will das Puzzle für zu Hause!} %Kommerzialisieurng
\question{Die Histogramme über die eigenen Daten einzuführen war erleuchtend} %Methodik, Aktivieurng
\question{Ich empfand es als gut, dass die Gruppe zum Mitdenken angeregt wurde } 
\question{Ich konnte eigenständig arbeiten} 
\question{Die Methoden waren vielfältig}
\question{Die Atmosphäre war locker}
%&\emph{Comments on Poster Content:}
\\
\newpage
%----------------------
%3.
\setcounter{qnumber}{0}
\stepcounter{qnumber}
\heading{Datenanalyse \& Aufgaben}
\question{Die Rechenaufgabe hat geholfen Detektoren zu verstehen} 
\question{Ich brauchte bei der Rechenaufgabe Hilfekarten} %Wenn nicht, Differnzierung! 
\question{Ich wusste was in der Datenanalyse zu tun ist}
\question{Die Benutzeroberfläche der Datenanalyse war einfach zu bedienen}
\question{Ich hätte mir gewünscht zu sehen, wie das Programm der Datenanalyse funktioniert}
\question{Das Design der Materialien war schön} %Cogn. Load

\\

\setcounter{qnumber}{0}
\stepcounter{qnumber}
\heading{Erkenntnis}
\question{Ich habe heute viel gelernt}
\question{Ich war überfordert}
\question{Das Niveau war hoch}
\question{Mein Bild von Teilchenphysiker*innen hat sich verändert} \\ 
&Wie hat sich Ihre Meinung dazu durch die Veranstaltung verändert? (Auf Skala ein $\times$ setzen) \\ 
&~~~~~~~~~~~~~~~~~~~\begin{tikzpicture}
    \draw [--] (0,0) -- (10,0);
    \draw  (-0.2,-0.03) node {$\bullet$ } node [below]{positiv};
        \draw  (5,-0.03) node {$\bullet$ } node [below]{gar nicht};
                \draw  (10.2,-0.03) node {$\bullet$ } node [below]{negativ};
\end{tikzpicture}\\
\question{Ich kann meinen Eltern später erklären was $\Omega$-Baryonen sind und wie man sie nachweist }
\question{Ich hatte heute viel Spaß}
\question{Ich wurde heute für ein Physikstudium begeistert} \\ 
&Wie hat sich Ihre Meinung dazu durch die Veranstaltung verändert? \\ 
&~~~~~~~~~~~~~~~~~~~\begin{tikzpicture}
    \draw [--] (0,0) -- (10,0);
    \draw  (-0.2,-0.03) node {$\bullet$ } node [below]{positiv};
        \draw  (5,-0.03) node {$\bullet$ } node [below]{gar nicht};
                \draw  (10.2,-0.03) node {$\bullet$ } node [below]{negativ};
\end{tikzpicture}\\

\question{Ich empfehle diese Veranstaltung}
\\



\bottomrule
\end{longtable}
Ich identifiziere mich als... \hspace{0.7cm} $\Box$ weiblich \hspace{0.7cm} $\Box$ männlich \hspace{0.7cm} $\Box$ divers \hspace{0.7cm} $\Box$ Keine Angabe \\


\small \noindent \bf Das möchte ich unbedingt noch loswerden...\vspace{0.6cm}

\bottomrule\vspace{0.6cm}
\bottomrule\vspace{0.6cm}
\bottomrule\vspace{0.6cm}
\bottomrule\vspace{0.6cm}
\bottomrule\vspace{0.6cm}


Wir bedanken uns für Ihre Teilnahme am Projekt und wünschen für Sie und Ihren Werdegang alles Gute!\thispagestyle{plain}

\end{document}